\documentclass[a4paper]{article}
\usepackage[pdftex]{geometry}
\usepackage[utf8]{inputenc}

\usepackage{listings}
\usepackage{amssymb}
\usepackage{amsmath}
\usepackage{multicol}
\usepackage{courier}
\usepackage{fancyhdr}
\usepackage{enumitem}
\usepackage{hyperref}
\usepackage{lastpage}

\lstset{columns=fullflexible}
\lstset{basicstyle=\scriptsize\ttfamily}
\lstset{showspaces=false}
\lstset{showstringspaces=false}
%
\geometry{portrait}
\setlength{\topmargin}{-0.5in}
\setlength{\oddsidemargin}{-0.5in}
%\setlength{\evensidemargin}{-in}
\setlength{\columnsep}{1in}
\setlength{\columnseprule}{0.2pt}
\textwidth 185mm
\textheight 250mm

\pagestyle{fancy}
\lhead{Nizhny Novgorod Branch of NRU HSE (Doroshenko, Orlova, Zarkov)}
\chead{}
\rhead{\thepage\ of \pageref{LastPage}}
\lfoot{}
\cfoot{}
\rfoot{}

\usepackage{cmap}					% поиск в PDF
\usepackage{mathtext} 				% русские буквы в формулах
\usepackage[T2A]{fontenc}			% кодировка
\usepackage[utf8]{inputenc}			% кодировка исходного текста
\usepackage[english,russian]{babel}	% локализация и переносы

%%% Дополнительная работа с математикой
\usepackage{amsmath,amsfonts,amssymb,amsthm,mathtools} % AMS
\usepackage{icomma} % "Умная" запятая: $0,2$ --- число, $0, 2$ --- перечисление

%% Номера формул
\mathtoolsset{showonlyrefs=true} % Показывать номера только у тех формул, на которые есть \eqref{} в тексте.

%% Шрифты
\usepackage{euscript}	 % Шрифт Евклид
\usepackage{mathrsfs} % Красивый матшрифт

\newcommand{\includecpp}[2]{
	\subsection{#1}\
	\lstinputlisting[language=c++, breaklines=true]{#2}
}

\newcommand{\floor}[1]{\lfloor #1 \rfloor}

\begin{document}
	\thispagestyle{fancy}
	\pagenumbering{Roman}
	
	\begin{center}
		\Huge\textsc{ACM-ICPC Team Reference Document}
		
		\huge HSE-NN 2
		
		\vspace{0.50cm}
	\end{center}
	
	\begin{multicols*}{2}
		\tableofcontents
		
		\pagenumbering{arabic}
		\setcounter{page}{1}
		%\input{contents.tex}
	\end{multicols*}
	\enlargethispage*{\baselineskip}
	\pagebreak
	
	% == CONTENTS == %
	
	\pagenumbering{arabic}
	\setcounter{page}{2}
	%
	\begin{multicols*}{2}
		\section{Шаблон}
		\lstinputlisting[language=C++]{fcpp/template.cpp}
		\section{Алгоритмы на строки}
		\subsection{Префикс-функция}
		\lstinputlisting[language=C++]{fcpp/pfunc.cpp}
		\subsection{Z-функция}
		\lstinputlisting[language=C++]{fcpp/zfunc.cpp}
		\section{Алгоритмы на графах}
		\subsection{Алгоритм Дейкстры $O(n^2)$}
		was - брали вершину или нет \\
		v - список смежности \\
		d - массив расстояний для точки x
		\\
		\\
		\\
		\\
		\\
		\lstinputlisting[language=C++]{fcpp/dijkstra-n2.cpp}
		\subsection{Алгоритм Дейкстры $O(log(n) \cdot m)$}
		d - массив расстояний для точки x
		\lstinputlisting[language=C++]{fcpp/dijkstra-nm.cpp}
	\end{multicols*}
\end{document}